\documentclass[12pt]{elsarticle}
\usepackage{newtxtext}
\usepackage[margin=1in]{geometry}
\usepackage{natbib}
\bibliographystyle{abbrvnat}
\usepackage{color}
\usepackage{amsmath}
\usepackage{hyperref}
\usepackage{wrapfig}
\usepackage{titlesec}
\usepackage{xcolor,colortbl}


\begin{document}

\begin{center} \textbf{5. ENGAGEMENT AND COMMUNICATION PLAN} \end{center}

\section*{Advisory Board} 

Our project will be guided by a Fish Passage Policy Advisory Committee (FPPAC) (Table~\ref{tab:sab}) comprised of community collaborators, who are fish passage scientists and practitioners and leaders in \textcolor{blue}{equity, access, and community engagement}. At each phase of our engagement plan, outlined below, we will rely on advice from our Advisory Committee. Members of FPPAC will include individuals working directly in the area of barrier culvert restoration, individuals working with tribal nations impacted by fish passage in the Case Area, individuals with a history of engaging with our key stakeholder groups, and individuals generating science relevant to our study problem. The FPPAC already has committed representation from the Squaxin Island Tribe, the Tulalip Tribes, Washington State Recreation and Conservation Office, and the King County Fish Passage Restoration Program (see Letters of Commitment). At the time of the submission of this proposal, we are actively seeking to expand this group to representation from other tribes, barrier ownership entities, and stakeholders. 

In addition to community collaborators serving on our FPPAC we will collaborate with fish passage scientists at The Nature Conservancy and scientists at the National Oceanic and Atmospheric Administration involved in the creation of FISH\emph{Pass} (see Letters of Commitment).

\begin{table}[h]
\caption{Fish Passage Policy Advisory Committee (FPPAC) \label{tab:sab}}
\centering
\begin{tabular}{lcc}\hline
 Name & Job Title & Affiliation  \\\hline
Jeff Dickison& Assistant Natural Resources Director &  Squaxin Island Tribe\\
& & Natural Resource Center\\
\rowcolor[gray]{.9} Steve R Hinton &  Conservation Scientist&  Tulalip Tribes  \\
\rowcolor[gray]{.9}& &Treaty Rights \& Government Affairs\\
Marc Duboiski & Outdoor Grants Manager & Washington State Recreation\\
& & and Conservation Office\\
\rowcolor[gray]{.9}Dave Caudill & Outdoor Grants Manager & Washington State Recreation\\
\rowcolor[gray]{.9}& & and Conservation Office\\
Evan Lewis &  Fish Passage Projects Manager&  King County  \\
& & Fish Passage Restoration Program\\
\hline
\end{tabular}
\end{table}

\section*{Engagement activities}

In the beginning of YR1, in the initial phase of the project, we will organize a workshop, Workshop 1, intended to uncover the objectives and challenges in culvert barrier replacement for key user groups including representatives from relevant tribal nations, city and county governments, restoration funding agencies such as the Fish Barrier Removal Board, WDFW, and WSDOT. The workshops will begin with a presentation of our proposed framework and online tool as a straw-man proposal in order to generate discussion and elicit ideas on how to capture fundamental real-world priorities and constraints in barrier culvert removal. We will gather feedback during the workshops and through post-workshop surveys. Ideas coming from Workshop 1 will be incorporated into our project to the extent possible given data and computational limitations, which will be made clear to workshop participants.

In a second phase, at the beginning of YR2, we will organize a second workshop, Workshop 2, to present preliminary results (e.g.\ tradeoffs between various objectives) and demonstrate the functionality of three preliminary working versions of the online tool. One version of the tool will be based on the weighted objective function, another will be based on the constraint-based method, and the third version will be based on the Pareto Frontier approach. This second workshop will demonstrate how feedback from Workshop 1 was incorporated in our framework and tool and provide a more in-depth discussion of our data inputs to the tool, e.g.\ a demonstration of the quality of our cost estimates, and a visual demonstration of results from various optimization methods. Workshop 2 will provide stakeholders with a final opportunity to guide key features of the framework and solicit feedback on the usability of the online DST and the optimization approach perceived to be most intuitive. 

Finally, at the end of YR2, we will host a third workshop, Workshop 3, which will be an interactive workshop to launch our finalized online tool letting potential users directly engage with the tool. In preparation for this workshop, we will develop a video tutorial that will present content from the DST user guide in a way that is accessible and engaging. The workshop will begin with a screening of the video tutorial. Then, we will engage participants with exercises that highlight the tradeoffs between various objectives, gains from coordination, and how alternative budget/funding scenarios (defined by budget levels and their distribution across time) impact the culvert restoration packages with the highest return on investment. As with Workshop 1, participant feedback will be collected through post-workshop surveys.


\section*{Communication with Non-Scientific Audiences}


\end{document}