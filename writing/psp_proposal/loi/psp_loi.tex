\documentclass[12pt]{elsarticle}
\usepackage{amsmath}
\usepackage{times}
\usepackage{color}
\usepackage{amssymb}
\usepackage{natbib}
\bibliographystyle{unsrt}
\usepackage{hyperref}
\usepackage{setspace}
\usepackage[section]{placeins}
\usepackage{graphicx}
\usepackage{mathrsfs}
\usepackage{colortbl}


\usepackage[margin=1in]{geometry}
%\usepackage{fancyhdr}
\pagestyle{empty}
\usepackage{titlesec}
\usepackage[para]{footmisc}
\titleformat{\section}[runin]{\normalfont\bfseries}{\thesection.}{3pt}{}
%\lhead{}
%\chead{}
%\rhead{}
%\cfoot{}

\urlstyle{same} 
\begin{document}
	
\begin{center}
		\textbf{Multiobjective Decision Support Tools for Strategic Fish Passage Barrier Correction}
		
		(Integrated Socio-Ecological Systems; October 16, 2021 -- June 30, 2023)

		\vspace{0.1cm}
		Sunny Jardine \footnote{Lead PI (\url{jardine@uw.edu})} \footnote{University of Washington -- School of Marine \& Environmental Affairs}, Braeden Van Deynze \footnotemark[2], Mark Schueurell \footnote{University of Washington -- School of Aquatic \& Fisheries Sciences}, Robby Fonner \footnote{NOAA Fisheries -- Northwest Fisheries Science Center}, Dan Holland \footnotemark[4]
\end{center}

	
	\section{Project Overview}
	\subsection{Background} 
	In 2001, Washington state was sued by the United States Department of Justice on behalf of 21 Puget Sound and Western Washington tribes for violating treaty fishing rights. The plaintiffs argued that state-owned culverts restrict salmon and steelhead access to historical upstream spawning habitat, leading to declines in salmon abundance and violating the Stevens Treaties which guarantee a right to fish within relinquished tribal lands (referred to as the "Case Area") \citep{hickey_highway_2018}. The lawsuit resulted in a 2013 federal court injunction requiring the state remove barrier culverts under its jurisdiction such that 90\% of blocked fish habitat is made accessible by 2030 and the remainder be restored by the end of their life. After nearly two decades of legal battles, in 2018, the U.S. Supreme Court ruled in favor of the tribes, upholding the 2013 federal injunction. 
	
	Importantly, the 2013 injunction strictly applies to approximately 1,000 state-owned culverts whereas there exist an estimated 3,000 and 1,300 additional barrier culverts owned by counties and cities respectively, along with barrier culverts on private lands, often on the same streams as state-owned culverts \citep{brown_coming_2019}. Cities and counties are relatively resource constrained and depend on state and federal grant funds for barrier culvert removal and related strategic planning. 
	
	Currently WSDOT prioritizes barrier culverts for removal based on factors such as: amount of habitat blocked and habitat quality, tribal input, project cost, traffic detour management during construction, maintenance issues, partnership opportunities, the presence and number of ESA listed species, and permitting constraints. Counties, cities, and other culvert-ownership entities have developed their own prioritization frameworks with limited resources and data to inform prioritization. Strategic planning is therefore complicated by the dispersed responsibility for interconnected barriers, heterogeneous prioritization frameworks, and the cost of acquiring information needed for decision making. The presence of multiple barrier ownership entities within a single watershed means that the benefits of any one entity's culvert restoration actions depend on the culvert restoration actions of other actors and suggests potential gains from coordination. 
	
	While there is a rich academic literature applying optimization tools to fish passage \citep{ohanley_optimizing_2005, kuby_multiobjective_2005, mcmanamay_commonalities_2019, couto_safeguarding_2021}, only recently have these tools gained traction with resource managers, who have more frequently relied on ``rank and score'' methods due to their simplicity. Such methods have been criticized for failing to fully account for dependencies between barrier correction actions and budget considerations \cite{mckay_comparison_2020}. In recent years, examples have emerged of optimization tools made accessible to managers through an online decision support tool (DST) (i.e.\ FISH\emph{Pass} developed for California \citep{ohanley_optipass_2015} and Fishwerks developed for the Great Lakes \citep{moody_pet_2017}). However, no such tool yet exists to bring optimization methods to barrier correction prioritization in the Case Area. Our project will build such a tool, tailored to the specific needs of regional resource managers and stakeholders through a participatory engagement process. Expanding on existing barrier optimization methods, our tool will account for multiple objective types including risk management and distributional equity, allowing the exploration of potential trade-offs between policy goals. 
	
	\subsection{Alignment with Priority Science Actions} % 1/4 page
	
	Our project is most directly aligned with Priority Science Action 12: ``Refine risk assessment tools and scenario development and analyses to improve our understanding of highly uncertain, complex and inter-related challenges and solutions to provide information that can be used to identify actions to achieve a more resilient Puget Sound ecosystem.'' Coordinating culvert barrier correction across multiple agencies with distinct and limited budgets is inherently a complex challenge. Our project will develop a novel framework that will demonstrate trade-offs between potentially competing priorities shared with stakeholders via a on online DST. Our approach is consistent with a number of recommendations set forth by the Science Panel, including incorporation of indigenous knowledge, coordination of efforts across entities, communication targeted to appropriate audiences, and the evaluation of trade-offs between approaches.
	
	\subsection{Project Summary}\label{sec:summary} % 1 1/2 page
	% Emphasize 4 components of optimization: costs, habitat, equity, risk associated with each plan ==> requires submodels to compute for each barrier under each setting
	
	Our proposed research will develop a data-driven framework for project prioritization, within the Case Area, that synthesizes multiple geospatial datasets with statistical economic and ecological models to identify restoration plans that maximize ecological, social, and economic objectives at a given funding level. Our framework will be used to assess the trade-offs between key objectives (e.g.\ increasing salmon habitat, an equitable distribution of habitat gains, and mitigating investment risk) as well as gains from coordinating barrier culvert replacement across key entities (the state, counties, and cities) and alternative funding streams. We will make the data, models, and framework accessible to users through an open-source online DST. In what follows we describe our framework and DST in more detail.
	
	The utility of our framework and DST critically depends on how well real-world priorities and constraints are reflected. Thus, in the initial phase of the project we will organize a series of workshops where we elicit objectives and constraints from key user groups (see Section~\ref{sec:engage}). For purposes of illustration, here we describe potential factors to be included and the data and models to support their inclusion.   
	
	% Step 1: Cost estimates (add sentence on how cost estimates are used in existing planning tools?)
	Barrier removal is costly and managers are budget constrained. To incorporate this constraint, we will estimate the cost of culvert restoration for all known existing fish passage-blocking culverts within the Case Area. Culverts in need of restoration will be identified using the Washington State Fish Passage and Diversion Screening Inventory (FPDSI) database maintained by the Washington Department of Fish and Wildlife (WDFW). Cost estimates will be based upon predictive models developed from over 1,200 culvert projects completed between 2001 and 2015 documented in the Pacific Northwest Salmonid Habitat Projects (PNSHP) dataset \citep{katz_freshwater_2007,noauthor_pacific_2021}. Project records are linked with potential predictor variables, including stream slope (\%), bankfull width, road class, elevation, etc. Cost models are currently being finalized by our research team under an ongoing project. For the proposed project, we will leverage the datasets and code we have already developed to explore the predictive performance of several parametric and non-parametric statistical learning methods, selecting the methodology that provides superior out-of-sample predictive power in the injunction area.
	
	% Step 2: Benefit estimates 
	For each culvert restoration plan, defined as a combination of multiple culverts restored, we will quantify the expected increase in habitat quantity, measured as lineal distance, for the five species of Pacific salmon. Spatial dependence will drive restoration benefits, because the culvert restoration downstream determines the benefits from culvert restoration upstream. Estimated habitat quantity gains will be calculated using the USGS National Hydrography Dataset, NDHPlus High Resolution, and the WDFW FPDSI database, which contains information about fish species affected by culvert blockages. Habitat gains will be weighted by habitat quality metrics developed to account for habitat needs for each salmon species through their freshwater life stages. 
	
	%
	Equity is another important dimension to consider. A prioritization framework that simply maximizes expected salmon habitat given a budget constraint could potentially lead to a culvert restoration plan that only benefits a single user group. We will explore alternative equity strategies that prioritize restoration plans that provide an equitable distribution of habitat gains across user groups, e.g.\ across the 21 tribal nations involved in the 2001 lawsuit. We will explore various equity metrics, e.g.\ a Gini coefficient, using geospatial data on all salmon runs in the injunction area together with geospatial data delineating user groups, e.g.\ the usual and accustomed fishing areas for each tribe. 
	
	%
	Risk mitigation is yet another important factor to consider when selecting portfolios of barrier culverts to remove. Returns to investments in barrier culvert removal are risky driven by the possibility of low salmon returns to habitat, population extinction driven by both environmental shocks, and future human impacts including impacts to water quality through urbanization \citep{ettinger_prioritizing_2021}. Drawing from the literature on restoration portfolio diversification, we will estimate the degree to which the risk in returns to barrier culvert removal plans can be mitigated through diversification. We will explore risk metrics ranging in complexity from a simple measure of the spread of investments across all salmon runs with habitat in the injunction area to increasingly complex risk metrics that utilize information on the potential negative covariance in expected returns to multiple salmon populations to exploit opportunities for portfolio diversification \citep{sanchirico_empirical_2008, jardine_fishermen_2015, johnston_combining_2002}.  
	
	%Table \ref{tab:factors} summarizes potential factors to be included in our restoration planning framework. With these factor inputs, the framework will employ mixed integer linear programming to identify the optimal restoration package that satisfy the manager's budget and preferences over the various factors included in the prioritization. The framework will also account for the ownership of other culverts along the stream network, allowing managers to consider plans that correct (or are contingent on correction of) culverts owned by other parties.
	
	% Example of how might be used
	As an illustrative example, suppose Kitsap County wants to define a restoration plan (a package of culverts to be restored) that balances habitat increases for Chinook salmon in the injunction area with equity and risk mitigation. Further suppose Kitsap County had a budget of B dollars to invest in the restoration plan and does not want to restore any culverts outside of its jurisdiction. Our framework would solve the following problem (blue text represents manager inputs):
	\begin{equation*}
		\substack{\text{\large max}}_{\boldsymbol{c}}\hspace{0.25in} \textcolor{blue}{w_1}\:\: \text{miles of Chinook habitat} + \textcolor{blue}{w_2}\:\: \text{equity metric} + \textcolor{blue}{w_3}\:\: \text{risk mitigation metric},
	\end{equation*}
	\begin{equation*}
		\text{subject to:  total cost} \le \textcolor{blue}{B} \text{ and } \boldsymbol{c} \in \{\textcolor{blue}{\boldsymbol{c}_{Kitsap}}  \},
	\end{equation*}
	
	%\noindent and
	%
	%\begin{equation*}
	%\boldsymbol{c} \in \{\textcolor{blue}{\boldsymbol{c}_s}  \},
	%\end{equation*}
	
	where $\boldsymbol{c}$ are culverts included in the restoration plan, $\boldsymbol{c}_{Kitsap}$ are the subset of barrier culverts owned by Kitsap county, B is total amount of funding that can be spent, and $w_1-w_3$ are the weights that managers place on each objective. The problem will be additionally constrained so that benefits from upstream culvert removal cannot be captured without first removing downstream blockages.
	
	The problem will be solved using R, a free software environment that supports integer programming. The user interface to the prioritization framework, or DST, will be an online app created with the Shiny package for R and hosted on the Shiny Server. Similar tools have been built using various proprietary software programs \citep{ohanley_optipass_2015, moody_pet_2017, mcmanamay_commonalities_2019}. Through using an open-source optimization framework coupled with an open-source user interface we maximize accessibility and customizability.  
	
	\subsection{Expected Outcomes} % 1/2 page
	
	We anticipate two major research outcomes. First, our project will produce publicly-available, well-documented, open-source prioritization DST for barrier culvert removal in the Case Area. The DST can serve a coordinating function by providing a framework to evaluate restoration plans across various actors regardless of whether the restoration plan is one identified by our optimization framework. The DST can also support planning for cities and counties with limited access to data and quality cost estimates. Finally, the source code behind our open-source DST will be made widely available to users outside of the state of Washington to facilitate the adoption of and customization by a larger set of potential users. 
	
	Second, our research will form the basis of a student thesis, leading to one or more scientific publications, applying our optimization model to answer important research questions including:
	
	\begin{enumerate}
		\item How can distributional equity and project risk be accounted for in in fish passage problems? (Note, to our knowledge equity nor risk have been accounted for in the fish passage prioritization literature.)
		%\item How can project risk be accounted for in fish passage problems? (Note, to our knowledge risk has not been accounted for in the fish passage prioritization literature.)
		\item What are the gains in habitat, equity, and risk avoidance associated with coordination across actors (state agencies, local government, private landowners) and which of these multiple objectives is most affected by a lack of coordination across actors?
		%\item Who are winners and losers associated with low vs. high coordination settings?
		\item Where in Washington injunction area (sub-basins/watersheds) are culvert restoration plans associated with tradeoffs between potentially competing priorities (e.g., risk versus total habitat, equity versus total habitat), and where can "win-wins" occur (i.e., plans that meet multiple objectives without reducing others)?
	\end{enumerate}
	
	\subsection{Engagement Plan}\label{sec:engage}
	
	In the initial phase of the project, we will organize a series of workshops intended to uncover the objectives and challenges in culvert barrier replacement for key user groups including WSDOT, city and county governments, restoration agencies such as the Fish Barrier Removal Board, WDFW, and representatives from relevant tribal nations. The workshops will begin with a presentation of our proposed framework and online tool (described in Section~\ref{sec:summary}) as a straw-man proposal in order to generate discussion and elicit ideas on how to capture fundamental real-world priorities and constraints in barrier culvert removal. Ideas coming from the initial series of workshops will be incorporated into our project to the extent possible given data and computational limitations, which will be made clear to workshop participants.
	
	In a second phase, we will organize a second series of workshops to present preliminary results and demonstrate the functionality of a development version of the DST. This second series of workshops will demonstrate how feedback from the initial workshops was incorporated in our framework and tool and provide a more in-depth discussion of our data inputs to the tool, e.g.\ a demonstration of the quality of our cost estimates and a visual demonstration of our preliminary spatial definition of our equity metric. The second series of workshops will provide stakeholders with an opportunity to guide key features of the framework and solicit feedback on the usability of the online DST.
	
	In the final phase, we will host an interactive workshop to launch the DST, letting stakeholders directly engage with the tool. The workshop will begin with a screening of the video tutorial that will accompany the DST. Then, we will engage participants with exercises that highlight the tradeoffs between various objectives, gains from coordination, and how alternative budget/funding scenarios impact the culvert restoration packages with the highest return on investment. Throughout all three stages of our engagement plan, we envision relying on Puget Sound Partnership (PSP) communications resources to maximize outreach to our tool's target audiences and to structure our workshops based on previous experience with similar projects in the region. 
	
	At each phase, we will rely on advice from a Fish Passage Policy Advisory Committee. Members of our advisory committee will include individuals working directly in the area of barrier culvert restoration and those with a history of engaging with our key stakeholder groups. The recruitment of this committee is already underway, with commitments from resource managers from King County, the Tulalip and Squaxin Island Tribes, and state agencies. We envision also including members of PSP Boards, including the Ecosystem Coordination, Science Panel, and Puget Sound Salmon Recovery Council, on our advisory committee as their expertise and time permit, in order to better coordinate our efforts with other PSP projects priorities. 
	
	\section{DEI Statement}
	Our project addresses a demonstrated policy priority for tribal nations in the Puget Sound region: improving fish passage for salmon. Our optimization framework will be co-developed with input from tribal resource managers and communities to ensure that the resulting tools reflect the needs of indigenous communities. By including representation from tribal and non-tribal governments on our advisory committee and specifically targeting workshops to indigenous communities, our objective is to elevate their voices in fish passage policy. When engaging with indigenous communities, we commit to following best practices for researchers, including reaching agreements with indigenous partners on data ownership, co-authorship, and presentation of sensitive results to outside audiences \cite{ban_incorporate_2018}.
	
	\section{Budget}
	Budget includes two months of summer salary for PI Jardine, postdoctoral salary and benefits for co-PI Van Deynze for the project term, funding for one research technician who will lead the development of the Shiny app, and funding for one graduate research assistant in the second year of the project. \textit{Other} includes F\&A (indirect), funding for the execution of our engagement plan, and publication fees.
	
		\begin{table}[h]
				\caption{Draft Budget} 	
			\centering
				\begin{tabular}{ ccc } %\hline
			\hline
			 & Year 1  & Year 2  \\
			 & (Oct. 1, 2021 -- June 30, 2022) & (Oct. 1, 2022 -- June 30, 2023) \\
			 \hline
			\rowcolor[gray]{.9} Salary \& benefits &  \$112,618  & \$208,711 \\ 
			Travel & \$1,500 & \$1,500 \\
			%\rowcolor[gray]{.9} Sub-contracts & -- & -- \\ 
			%Supplies \& equipment & -- & -- \\
			\rowcolor[gray]{.9} Other & \$33,451 & \$75,277 \\
			\hline
			\textit{Annual total} & \$147,569 & \$285,488 \\
			\hline
			& \textbf{\textit{Grand total}} & \$433,057 \\
			\hline
		\end{tabular}
	\end{table}

	
	% A table would do here
	% RFP requests breakdown per state fiscal year (July 1 - June 30) for following
	% Salary & Benefits
	% Subcontracts (none)
	% Travel
	% Supplies and equipment (none)
	% Other (a few grand for publishing fees)
	
	
	\clearpage
	\footnotesize
	\bibliography{psp}
\end{document}