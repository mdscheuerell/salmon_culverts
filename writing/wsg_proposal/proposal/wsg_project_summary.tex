\documentclass[12pt]{elsarticle}
\usepackage{newtxtext}
\usepackage[margin=1in]{geometry}
%\pagestyle{empty}
\usepackage{color}
\usepackage{amsmath}
\usepackage{hyperref}
\usepackage{wrapfig}
\usepackage{titlesec}
\usepackage{colortbl}
%\titleformat{\section}[runin]{\normalfont\bfseries}{\thesection.}{3pt}{}


\begin{document}
	
	\begin{center} \textbf{PROJECT SUMMARY} \end{center}
	
	%
		\subsection*{Goals}
\begin{enumerate}
\item To contribute our expertise and knowledge in economics, optimization, and conservation biology to maximizing the returns on investment from barrier culvert restoration in Washington State.
\item To learn about the diverse priorities, objectives, and concerns of tribes and various other stakeholders as relates to barrier culvert restoration in Washington State.
\item To train graduate students in producing high-quality, relevant, and reproducible applied research related to natural resources management.
\end{enumerate}

\subsection*{Objectives}
\begin{enumerate}
\item Catalog methods and datasets used in generating fish passage prioritization indices for all counties, and any other entities using prioritization indices, within the Washington State Case Area. 
\item Generate consistent predicted cost estimates for all barriers to fish passage within the Washington State Case Area. 
\item Generate consistent habitat quality metrics associated with all barriers to fish passage within the Washington State Case Area for all five species of Pacific salmon and steelhead.
\item Develop a data-driven optimization framework for project prioritization, within the Washington State Case Area, that synthesizes multiple geospatial datasets with statistical economic and ecological models, and incorporates feedback based on multiple workshops, to identify restoration plans that maximize ecological, social, and economic objectives at a given funding level.
\item Develop a customizable open-source online decision support tool (DST), along with a video tutorial, to make our optimization framework accessible to tribes, stakeholders, managers, and academics. 
\end{enumerate}
			% Copy-paste from the narrative.
			% 1603 / 2000 character limit
			
		\section{Methodology} 
			% Succinctly describe the methods and approach to be used in accomplishing the objectives. 
			\begin{enumerate}
				\item We will catalog all methods and datasets used for fish passage prioritization within the Case Area. This catalog will allow for comparison across jurisdictions and serve as a starting point for considering data inputs into the optimization framework and DST.
				\item We will refine and apply predictive cost models currently being finalized by our team. We will generate consistent cost estimates for all inventoried barriers in the Case Area using statistical learning methods designed to maximize out-of-sample predictive power.
				\item We will use statistical models to relate measures of salmon habitat quality to features of the environment, building upon these earlier efforts to develop a consistent metric of salmon habitat quality based on instream temperature and flow data, as well as upland features related to riparian forest density and composition, road density, elevation, and watershed area.
				\item We will develop a novel optimization framework for identifying cost-effective restoration plans that meet multiple planning
				goals defined in our workshop series, e.g. salmon habitat gains, equity considerations, and risk mitigation, under a fixed budget, allowing for the identification of fish passage restoration plans associated with tradeoffs or "win-wins" across objectives.
				\item We will build and deploy a web-based DST that will allow resource managers and other users to explore and compare fish passage restoration plans generated by our optimization framework. Users will be able to compare optimal and user-selected plans in terms of objectives (e.g.\ habitat miles, equity, risk) and how relative performance is affected by different funding levels or levels of coordination between barrier owners.
				\item Throughout our project we will conduct a series of three workshops with representation from tribes, resource managers, and other stakeholders. Through these workshops we will refine our methods and DST to ensure that the resulting tool reflects the various objectives and practical constraints to fish passage restoration in Washington State.
			\end{enumerate}
			% 1988 / 2000 character limit
		\section{Rationale}
			% Concisely state the problem or opportunity addressed. Indicate why the project is important, appropriate for WSG support, and why the proposed approach is necessary. Identify the expected outcomes of the project and potential project users.
			Across Western Washington, thousands of poorly-designed culverts at road crossings prevent migratory salmon from accessing potential habitat, hampering recovery efforts for declining populations. In 2013, a federal court found that barrier culverts on Washington State roads violate tribal treaty rights and issued an injunction requiring their replacement. While the injunction only applies to state-owned culverts with the ``Case Area'', thousands of additional barrier culverts are owned by local governments and private landowners in the Case Area, often within the same watersheds leading to inter-dependencies between barrier correction activities. Currently counties and other actors are largely acting independently to correct barriers to fish passage and there are differences across counties, e.g.\, in terms of goals, priorities, and resources for removing barriers to fish passage. 
			
			The State Legislature currently provides funding for projects addressing barriers to fish passage based on recommendations from the Brian Abbot Fish Barrier Removal Board. The board seeks utilizes diverse metrics provided by counties and other actors to determine priority projects. There is demand from both the Legislature and grantees to better define the Board's selection criteria for potential projects and overall prioritization strategy. Thus, there is a great need for science-based prioritization tools that are consistent across all barriers to fish passage, regardless of ownership and that account for the diverse objectives of tribes, barrier ownership entities, county and state resource managers, and other stakeholder groups
			
			Our project will develop a consistent, data-driven framework for prioritizing fish passage barriers over multiple objectives, drawing from a rich literature on fish passage restoration plans that maximize return on investment, experiences from similar frameworks and tools developed to aid fish passage investments in California and the Great Lakes region, and our workshop series. 
			
			The decision support tool we develop will serve a coordinating function between barrier owners and managers by allowing the consistent comparison of alternative barrier correction plans. Users will be able to identify cost-effective plans that meet their specific user-defined objectives and budgets while considering opportunities to coordinate with other barrier owners. Our decision support tool will add to and complement ongoing efforts to restore fish passage in the state of Washington.
			% 1900 / 2000 character limit
			
			
			
\end{document}